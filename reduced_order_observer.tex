\documentclass[letterpaper,11pt]{article}
\usepackage[round]{natbib}
\usepackage[margin=1in,centering]{geometry}
\usepackage{amsmath}
\usepackage{amssymb}
\usepackage[pdftex]{hyperref}
\hypersetup{
    pdftitle={Bicycle reduced order observer},
    pdfauthor={Dale L. Peterson, Oliver Z. Lee, Mont Hubbard },
    pdfsubject={Bicycle roll angle estimation},
    pdfkeywords={bicycle, dynamics, control}}

\begin{document}
\title{Implementation and design of a bicycle roll angle estimator using a reduced order observer}
\author{Dale L. Peterson, Oliver Z. Lee, Mont Hubbard}
\date{\today}
\maketitle

%\begin{abstract}
We present the design and implementation of a roll angle estimator based on the
Whipple bicycle model with measurements of steer, roll rate,
steer rate, and steer torque.  The observer is a
linear system with one state, four inputs, and one output.  We discuss how the
speed dependent dynamics of the bicycle are taken into account by speed
dependent gain scheduling.
%\end{abstract}

%\section{Introduction}
A common requirement in control system design is the knowledge of the plant
states to be controlled. If the plant is observable through the available
measurements, an estimate of those states can be made and used in place of the
true states when applying the feedback control law. If some states are
directly measurable, it is possible design a reduced order
observer~\citep{Bryson1970} with fewer states than that of the plant (a
full order observer has the same number of states as the plant).
%From a design and implementation perspective, this can be advantageous because
%there may be fewer design parameters (thereby making the design simpler and
%hopefully easier), fewer filter s

The Whipple bicycle model states are lean $\phi$, steer $\delta$, lean
rate $\dot{\phi}$, steer rate $\dot{\delta}$.  Of these, the most
difficult to measure directly is the roll angle $\phi$. Techniques to measure
or estimate $\phi$ include optical sensors on both sides of the rear wheel axle to
measure the distance from the axle to the ground, mechanical trailers measuring
roll with a potentiometer, and IMU based solutions which employ rate
gyroscopes and/or accelerometers and Kalman filtering techniques to obtain
orientation estimates\cite{Boniolo2008}.

We have built a robotic bicycle equipped with an optical encoder to measure
steer $\delta$ and a rate gyroscope to measure roll rate $\dot{\phi}$.  The steer
rate $\dot{\delta}$ is computed using a simple low pass filtered derivative
($\frac{sa}{s+a}$). Using state space matrices determined from measurements
of bicycle physical parameters and the presumed Whipple model, we designed a
gain scheduled LQR controller which assumes full state feedback.  For the
purposes of the LQR controller, the states were taken to be the roll estimate
$\hat{\phi}$ and the three direct measurements ($\delta, \dot{\phi},
\dot{\delta}$) of the state.  We omit the details of the LQR gain selection,
and present the design of the reduced order observer.

%\section{Methods}
With the bicycle state $x = \left[\phi, \delta, \dot{\phi},
\dot{\delta}\right]^T$, input steer torque $T_\delta$, and measurements $z =
[\delta, \dot{\phi}, \dot{\delta}]^T$,
the linear state space bicycle equations are of the form:
\begin{equation*}
\dot{x} =\left[\begin{smallmatrix}0 & 0 & 1 & 0\\0 & 0 & 0 & 1\\a_{20} & a_{21} &
a_{22} & a_{23}\\a_{30} & a_{31} & a_{32} & a_{33}\end{smallmatrix}\right] x +
\left[\begin{smallmatrix}0\\0\\b_{20}\\b_{30}\end{smallmatrix}\right] T_\delta
\qquad
z = \left[\begin{smallmatrix}0 & 1 & 0 & 0\\ 0 & 0 & 1 & 0\\ 0 & 0 & 0 &
1\end{smallmatrix}\right] x
\end{equation*}
Where we have assumed we can directly measure steer $\delta$, roll rate
$\dot{\phi}$, and steer rate $\dot{\delta}$.  It is worth noting that the
$a_{20}$ and $a_{30}$ entries of the system dynamics matrix are independent of
forward speed, $a_{21}$ and $a_{31}$ depend on the square of forward speed, and
the remaining $a_{ij}$ depend linearly on forward speed.  Following
\citet*{Bryson1970}, we introduce a change of variables
\begin{align*}
\left[\begin{smallmatrix}w \\ z\end{smallmatrix}\right] &=
\left[\begin{smallmatrix}k_0 & k_1 & k_2 & k_3 \\ 0 & 1 & 0 & 0\\ 0 & 0 & 1 & 0\\ 0 & 0 & 0 &
1\end{smallmatrix}\right] x  \quad\implies\quad
x =
\left[\begin{smallmatrix}\frac{1}{k_{0}} & - \frac{k_{1}}{k_{0}} & -
  \frac{k_{2}}{k_{0}} & - \frac{k_{3}}{k_{0}}\\0 & 1 & 0 & 0\\0 & 0 & 1 & 0\\0
  & 0 & 0 & 1\end{smallmatrix}\right]\left[\begin{smallmatrix} w \\ z\end{smallmatrix}\right]
\end{align*}
where $k_0\ne0$.  From this change of variables, an observer for $w$ can be
synthesized as
\begin{align*}
\dot{\hat{w}} &= \frac{a_{20} k_{2} + a_{30} k_{3}}{k_{0}} \hat{w}
 + \left(a_{21} k_{2} + a_{31} k_{3} - \frac{k_{1} \left(a_{20} k_{2} + a_{30} k_{3}\right)}{k_{0}}\right) \delta \\
 &+ \left(a_{22} k_{2} + a_{32} k_{3} + k_{0} - \frac{k_{2} \left(a_{20} k_{2} + a_{30} k_{3}\right)}{k_{0}}\right) \dot{\phi}
 + \left(a_{23} k_{2} + a_{33} k_{3} + k_{1} - \frac{k_{3} \left(a_{20} k_{2} + a_{30} k_{3}\right)}{k_{0}}\right) \dot{\delta} \\
 &+ \left(b_{20} k_{2} + b_{30} k_{3}\right) T_\delta
\end{align*}

To stabilize the observer state equation we must choose $k_0, k_2, k_3$ such
that
\begin{equation*}
\frac{a_{20} k_{2} + a_{30} k_{3}}{k_{0}} < 0
\end{equation*}

The selection of $k_0$, $k_2$, and $k_3$, is guided by the control systems
design principle which suggests that observer poles be placed 3-10 times faster
than the fastest pole of the controlled plant. Since $a_{20}$ and $a_{30}$ are
independent of speed, the estimator eigenvalues can be arbitrarily assigned by
selection of fixed $k_0$, $k_2$, and $k_3$ that are independent of speed.

The selection of the $k_i's$ can be performed using several techniques, but all
techniques essentially distill down to how the plant model (and the associated
measurements of its physical) is trusted in comparison to the state
measurements. In the full paper to be submitted, we describe in detail the
design procedure of selecting the $k_i's$ to yield a roll angle observer with
good performance characteristics and desirable noise rejection.  We present
experiments of a robot bicycle with an implementation of this reduced order
observer to estimate roll $\phi$ and use the estimated roll and state
measurements to perform real time estimation and feedback control.  Design
considerations and directions for future refinements are also presented.

%However, the coefficients which multiply the measured states $z$ and the known
%input $T_\delta$ do depend upon speed.  meausured states $z = [\delta$ and the
%plant input .  Beyond this essential stability property, it is not entirely
%clear to me how the entries of $K$ should be selected to determine the
%coefficients that multiply the state measurements and steer torque
%input.

%\section{Conclusion}


\bibliography{library}   % name your BibTeX data base
\bibliographystyle{plainnat}      % mathematics and physical sciences

\end{document}
